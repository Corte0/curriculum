%%%%%%%%%%%%%%%%%%%%%%%%%%%%%%%%%%%%%%%%%
% Twenty Seconds Resume/CV
% LaTeX Template
% Version 1.1 (8/1/17)
%
% This template has been downloaded from:
% http://www.LaTeXTemplates.com
%
% Original author:
% Carmine Spagnuolo (cspagnuolo@unisa.it) with major modifications by 
% Vel (vel@LaTeXTemplates.com)
%
% License:
% The MIT License (see included LICENSE file)
%
%%%%%%%%%%%%%%%%%%%%%%%%%%%%%%%%%%%%%%%%%

%----------------------------------------------------------------------------------------
%	PACKAGES AND OTHER DOCUMENT CONFIGURATIONS
%----------------------------------------------------------------------------------------

\documentclass[a4paper]{twentysecondcv} % a4paper for A4

%----------------------------------------------------------------------------------------
%	 PERSONAL INFORMATION
%----------------------------------------------------------------------------------------

% If you don't need one or more of the below, just remove the content leaving the command, e.g. \cvnumberphone{}

\profilepic{foto.jpg} % Profile picture

\cvname{Cortesini Luciano} % Your name
\cvjobtitle{Estudiante de ingeniería electrónica} % Job title/career

\cvdate{22 de Octubre de 2003} % Date of birth
\cvaddress{J. Lascano Colodrero 2592} % Short address/location, use \newline if more than 1 line is required
\cvnumberphone{+54 351 7411768} % Phone number
\cvsite{http://en.wikipedia.org} % Personal website
\cvmail{cortesiniluciano@gmail.com} % Email address

%----------------------------------------------------------------------------------------

\begin{document}

%----------------------------------------------------------------------------------------
%	 ABOUT ME
%----------------------------------------------------------------------------------------

%\aboutme{Apasionado por la tecnología y la resolución de problemas, tengo experiencia en el desarrollo de proyectos técnicos. Busco siempre la mejora continua en mi trabajo y disfruto del desafío de aprender nuevas habilidades que me permitan aportar valor a los proyectos en los que participo.} % To have no About Me section, just remove all the text and leave \aboutme{}

%----------------------------------------------------------------------------------------
%	 SKILLS
%----------------------------------------------------------------------------------------

\cvskills{\fontsize{10}{15}\selectfont{Experiencia en el desarrollo de aplicaciones de consola y sistemas más complejos en C++. 

Conocimientos básicos en Python para desarrollo de scripts y tareas de automatización.

Experiencia en la gestión de repositorios Git para proyectos individuales y en equipo.

Manejo de software CAD para modelado 3D paramétrico.

Soy una persona adaptable, con un pensamiento analítico que me permite abordar y resolver problemas.}}

%----------------------------------------------------------------------------------------

%----------------------------------------------------------------------------------------
%	 IDIOMAS
%----------------------------------------------------------------------------------------

\skills{{Inglés/4},{Español/6}}


%----------------------------------------------------------------------------------------
%	 HOBBIES
%----------------------------------------------------------------------------------------

\cvhobbies{

Handball

Impresión 3D

Fotografía

Astronomía

Computación}


\makeprofile % Print the sidebar

%----------------------------------------------------------------------------------------
%	 EXPERIENCE
%----------------------------------------------------------------------------------------
\section{Perfil}

\fontsize{10}{15}\selectfont{Apasionado por la tecnología y la resolución de problemas, tengo experiencia en el desarrollo de proyectos técnicos. Busco siempre la mejora continua en mi trabajo y disfruto del desafío de aprender nuevas habilidades que me permitan aportar valor a los proyectos en los que participo.\\}

\section{Experiencia Laboral}

\begin{twenty} % Environment for a list with descriptions
	\twentyitem{03/2023 - 03/2024}{OMIXOM SRL}{TÉCNICO}{Ensamblado, configuración y mantenimiento técnico de estaciones meteorológicas.

    Desarrollo de proyectos internos para optimizar procesos y soluciones técnicas, empleando herramientas como Git, C++ y SolidWorks.}\\

	\twentyitem{06/2022 - 09/2022}{DESCAR ARGENTINA}{PASANTE}{Realización de relevamientos y documentación de circuitos eléctricos utilizando Solid Edge Electrical.

    Diseño y modelado 3D en Solid Edge para la creación de piezas y ensamblajes.}
\end{twenty}

%----------------------------------------------------------------------------------------
%	 EDUCATION
%----------------------------------------------------------------------------------------

\section{Educación}

\begin{twenty} % Environment for a list with descriptions
	\twentyitemst{2016-2022}{Técnico en electrónica}{Instituto Técnico Salesiano Villada}
{
  Promedio de 8.05\\
  \textbf{Logros académicos}
  \setlength{\leftmargini}{1em}
  \begin{itemize}
    \item Integrante del grupo ganador de la competencia CANSAT 2022
    \item Participación en ONIET:\\
          Electrónica discreta (Oro 2022)\\
          Microcontroladores (Plata 2022)\\
          Ciencias básicas (Plata 2021)
  \end{itemize}
}\\
	\twentyitemst{2023-actualidad}{Ingeniería en electrónica}{UTN-FRC}{Segundo año aprobado\\ Promedio (con aplazos) de 8,77}
\end{twenty}

%----------------------------------------------------------------------------------------
%	 PUBLICATIONS
%----------------------------------------------------------------------------------------

\section{Proyectos}

\begin{twenty} % Environment for a short list with no descriptions
	\twentyitemst{2020-actualidad}{FP3D}{Diseño, impresión 3d y servicio técnico}{Brindo servicios personalizados de diseño e impresión 3D, así como mantenimiento y reparación de impresoras 3D.}\\
	\twentyitemst{2024-actualidad}{HEXSAR}{Desarrollador de software}{Formo parte del equipo de desarrollo del proyecto Hexapod for Search and Rescue en el Centro de Investigación en Informática para la Ingeniería (CIII) de la UTN-FRC. Mi trabajo se enfoca en el uso de Linux, Python y GitLab.}
\end{twenty}

%----------------------------------------------------------------------------------------
%	 AWARDS
%----------------------------------------------------------------------------------------

%\section{Awards}
%
%\begin{twentyshort} % Environment for a short list with no descriptions
%	\twentyitemshort{1987}{All-Time Best Fantasy Novel.}
%	\twentyitemshort{1998}{All-Time Best Fantasy Novel before 1990.}
%	%\twentyitemshort{<dates>}{<title/description>}
%\end{twentyshort}

%----------------------------------------------------------------------------------------
%	 OTHER INFORMATION
%----------------------------------------------------------------------------------------

%\section{Other information}
%
%\subsection{Review}
%
%Alice approaches Wonderland as an anthropologist, but maintains a strong sense of noblesse oblige that comes with her class status. She has confidence in her social position, education, and the Victorian virtue of good manners. Alice has a feeling of entitlement, particularly when comparing herself to Mabel, whom she declares has a ``poky little house," and no toys. Additionally, she flaunts her limited information base with anyone who will listen and becomes increasingly obsessed with the importance of good manners as she deals with the rude creatures of Wonderland. Alice maintains a superior attitude and behaves with solicitous indulgence toward those she believes are less privileged.

%----------------------------------------------------------------------------------------
%	 SECOND PAGE EXAMPLE
%----------------------------------------------------------------------------------------

%\newpage % Start a new page

%\makeprofile % Print the sidebar

%\section{Other information}

%\subsection{Review}

%Alice approaches Wonderland as an anthropologist, but maintains a strong sense of noblesse oblige that comes with her class status. She has confidence in her social position, education, and the Victorian virtue of good manners. Alice has a feeling of entitlement, particularly when comparing herself to Mabel, whom she declares has a ``poky little house," and no toys. Additionally, she flaunts her limited information base with anyone who will listen and becomes increasingly obsessed with the importance of good manners as she deals with the rude creatures of Wonderland. Alice maintains a superior attitude and behaves with solicitous indulgence toward those she believes are less privileged.

%\section{Other information}

%\subsection{Review}

%Alice approaches Wonderland as an anthropologist, but maintains a strong sense of noblesse oblige that comes with her class status. She has confidence in her social position, education, and the Victorian virtue of good manners. Alice has a feeling of entitlement, particularly when comparing herself to Mabel, whom she declares has a ``poky little house," and no toys. Additionally, she flaunts her limited information base with anyone who will listen and becomes increasingly obsessed with the importance of good manners as she deals with the rude creatures of Wonderland. Alice maintains a superior attitude and behaves with solicitous indulgence toward those she believes are less privileged.

%----------------------------------------------------------------------------------------

\end{document} 

